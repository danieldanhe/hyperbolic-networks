% !TEX TS-program = lualatex

\documentclass[citation=project.bib,handout]{presentation}

\title{Networks and hyperbolic geometry}
\subtitle{Advanced Topic Post-Euclidean Geometry \\ Project Presentation}
\author{Daniel He \myname}
\institute{Singapore American School}
\hypersetup{pdfauthor = {Daniel He 賀守愚}}

\DeclareMathOperator\arcosh{arcosh}

\usepackage{qrcode}

\begin{document}

\frame{\titlepage}

\begin{frame}[standout]
  Welcome.

  \bigskip

  \qrcode[hyperlink,height=4cm]{https://github.com/danieldanhe/hyperbolic-networks}

  \medskip

  \textmd{Code available on GitHub at \\ \url{https://github.com/danieldanhe/hyperbolic-networks}}
\end{frame}

\frame{\tableofcontents}

\section{Networks}

\begin{frame}{What is a network?}
  A system of interconnected elements (called \imp{nodes} or \imp{vertices}) linked by relationships (called \imp{edges} or \imp{connections}). \pause

  \begin{example}
    \begin{multicols}{3}
      \begin{itemize}[<+->]
        \parskip=0em
        \item Sites on the Internet
        \item Social connections (“six degrees of separation”)
        \item Neural networks and genetic or protein interactions
        \item Road, railway, and air routes
        \item Food webs in ecosystems
        \item Citations among academic papers
        \item International alliances
        \item Finances among banks and institutions
        \item Co-purchases on e-commerce platforms
        \item Microservice dependencies in tech
        \item Transmission of infectious diseases
        \item Semantic networks
        \item Collaborations in arts, music, and film
        \item Blockchain and cryptocurrency transactions
      \end{itemize}
    \end{multicols}
  \end{example}
\end{frame}

\begin{frame}{Properties of these networks}
  What do all of these networks have in common? \pause

  They cannot be modelled by completely random networks, where connections are made by uniform random chance. There are two main differences: \pause
  \begin{itemize}
    \item \imp{Heterogeneity}. \pause Some nodes have more connections than others. \pause On social media, most people have a few hundred followers, but a few celebrities have millions. \pause
    \item \imp{Clustering}. \pause Connections within groups tend to happen more than outside of them. \pause If A and B follow each other, and B and C follow each other, A and C likely follow each other too.
  \end{itemize}
\end{frame}

\begin{frame}{Heterogeneity and clustering}
  \begin{tabular}{lll}
    \toprule
    & Low heterogeneity & High heterogeneity \\ \midrule
    Low clustering & \makecell{\input{generate-points/low_clust_low_hetero.tikz}} & \makecell{\input{generate-points/low_clust_high_hetero.tikz}} \\
    High clustering & \makecell{\input{generate-points/high_clust_low_hetero.tikz}} & \makecell{\input{generate-points/high_clust_high_hetero.tikz}} \\ \bottomrule
  \end{tabular}

  All networks shown here have approximately \(60\) nodes and \(120\) connections.
\end{frame}

\begin{frame}{Scale-free networks}
  We will look at \imp{scale-free} networks specifically. The degree distribution follows a power law. \pause

  The fraction \(P(k)\) of nodes having \(k\) connections is approximately
  \[P(k)\sim k^{-\gamma}\]
  for large \(k\). \pause

  Typically, \(2<\gamma<3\). \pause For the Internet, \(\gamma\approx2.1\).\footcite{Krioukov_2010}
\end{frame}

\begin{frame}{The solution: hyperbolic geometry}
  \begin{tabular}{lll}
    \toprule
    & Low heterogeneity & High heterogeneity \\ \midrule
    Low clustering & \onslide<2->{Random graphs\footcite[Connects pairs completely at random.]{Erdős_Rényi_1959}} & \onslide<3->{Preferential attachment\footcite[New nodes prefer to connect to already well-connected hubs.]{Barabási_Albert_1999}} \\
    High clustering & \onslide<4->{Small-world\footcite[Starts with a regular, clustered lattice and rewires a few connections to create long-range shortcuts.]{Watts_Strogatz_1998}} & \onslide<5->{Hyperbolic geometry} \\ \bottomrule
  \end{tabular}
\end{frame}

\section{Hyperbolic geometry}

\begin{frame}{The three geometries}
  \begin{tabular}{llll}
    \toprule
    & Hyperbolic & Euclidean & Spherical \\ \midrule
    \onslide<+->{Model & \makecell{\begin{tikzpicture}
      \begin{axis}[
        ticks=none,
        height=4cm,
        width=4cm,
        xmin=-1.5,
        xmax=1.5,
        ymin=-1.5,
        ymax=1.5,
        zmin=-1.5,
        zmax=1.5,
        axis equal,
        clip=true
      ]
        \addplot3[
          surf,
          fill opacity=0.15,
          fill=white,
          samples=40,
          domain=0:360,
          y domain=-7.5:7.5
        ] ({sqrt(1+y^2)*cos(x)}, {sqrt(1+y^2)*sin(x)}, {y});
        \draw[->, thick, white] (0,0,0) -- (1,0,0) node[midway, above] {\(R\)};
      \end{axis}
    \end{tikzpicture}} & \makecell{\begin{tikzpicture}
      \begin{axis}[
        ticks=none,
        height=4cm,
        width=4cm,
        xmin=-1,
        xmax=1,
        ymin=-1,
        ymax=1,
        zmin=-1,
        zmax=1,
        axis equal,
        clip=true
      ]
        \addplot3[
          surf,
          fill opacity=0.15,
          fill=white,
          samples=40,
          domain=-6:6,
          y domain=-6:6
        ] {0};
      \end{axis}
    \end{tikzpicture}} & \makecell{\begin{tikzpicture}
      \begin{axis}[
        ticks=none,
        height=4cm,
        width=4cm,
        xmin=-1,
        xmax=1,
        ymin=-1,
        ymax=1,
        zmin=-1,
        zmax=1,
        axis equal,
        clip=false
      ]
        \addplot3[
          surf,
          fill opacity=0.15,
          fill=white,
          samples=40,
          domain=0:180,
          y domain=0:360
        ] ({cos(y)*sin(x)}, {sin(y)*sin(x)}, {cos(x)});
        \draw[->, thick, white] (0,0,0) -- (1,0,0) node[midway, above] {\(R\)};
      \end{axis}
    \end{tikzpicture}}} \\
    \onslide<+->{\imp{Curvature \(K\)} & \(=-1/R^2<0\) & \(=0\) & \(=+1/R^2>0\)} \\
    \onslide<+->{\makecell[l]{Parallel lines \\ (\imp{geodesics})} & \(\infty\) & \(1\) & \(0\)} \\
    \onslide<+->{\makecell[l]{Sum of angles \\ in triangle} & \(<\pi\) & \(=\pi\) & \(>\pi\)} \\
    \onslide<+->{\makecell[l]{Circumference of \\ a circle} & \(>2\pi r\) & \(=2\pi r\) & \(<2\pi r\)} \\ \bottomrule
  \end{tabular}
\end{frame}

\begin{frame}{Distance and area behaviour}
  \onslide<1->{Consider a circle of radius \(r\) on a plane with radius of curvature \(R\).}

  \onslide<2->{
    \begin{tabular}{lll}
      \toprule
      & Circumference of a circle & Area of a circle \\ \midrule
      \onslide<7->{Hyperbolic & \(C_{\mathrm H}(r)=2\pi R\sinh\Bigl(\dfrac rR\Bigr)\onslide<8->{\sim e^{r/R}}\) & \(A_{\mathrm H}(r)=2\pi R^2\biggl(\cosh\Bigl(\dfrac rR\Bigr)-1\biggr)\onslide<8->{\sim e^{r/R}}\)} \\
      \onslide<3->{Euclidean & \(C_{\mathrm E}(r)=2\pi r\onslide<4->{\sim r}\) & \(A_{\mathrm E}(r)=\pi r^2\onslide<4->{\sim r^2}\)} \\
      \onslide<5->{Spherical & \(C_{\mathrm S}(r)=2\pi R\sin\Bigl(\dfrac rR\Bigr)\onslide<6->{\to0}\) & \(A_{\mathrm S}(r)=2\pi R^2\biggl(1-\cos\Bigl(\dfrac rR\Bigr)\biggr)\onslide<6->{\to4\pi R^2}\)} \\ \bottomrule
    \end{tabular}
  }

  \onslide<9->{Hyperbolic geometry has “more space”.}
\end{frame}

\begin{frame}{The Poincaré disc}
  \onslide<1->{How do we represent an infinite, expanding space in a way we can visualize and compute with?}

  \begin{itemize}
    \onslide<2->{\item The \imp{Poincaré disc} is a mapping of a hyperbolic plane (assume onto a unit disc with \(K=-R=-1\)); it preserves angles but not distances.}
    \onslide<3->{\item The distance between two points is
    \[d(\bm u,\bm v)=\arcosh\Bigl(1+\frac{2\lVert\bm u-\bm v\rVert^2}{(1-\lVert\bm u\rVert^2)(1-\lVert\bm v\rVert^2)}\Bigr)\]}
    \onslide<5->{\item Geodesics are circular arcs perpendicular to the boundary.}
  \end{itemize}

  \onslide<4->{
    \begin{example}
      \(\{3,7\}\)\quad
      \includegraphics[height=2cm]{tilings/tiling-3-7.pdf}
      \qquad
      \(\{4,5\}\)\quad
      \includegraphics[height=2cm]{tilings/tiling-4-5.pdf}
      \qquad
      \(\{5,4\}\)\quad
      \includegraphics[height=2cm]{tilings/tiling-5-4.pdf}
    \end{example}
  }
\end{frame}

\section{The connection between the two}

\begin{frame}{Network generation in hyperbolic space}
  \begin{itemize}
    \onslide<1->{\item There are \(N\) nodes. Each node \(i\) is assigned coordinates \((r_i,\theta_i)\).}
    \begin{itemize}
      \item \onslide<2->{\(r_i\) is drawn from the probability density function
      \[\rho(r)=\frac{\alpha\sinh(\alpha r)}{\cosh(\alpha L)-1}\sim e^{\alpha r}\]
      where \(\alpha\) and \(L\) are chosen such that \(\bar k=\int_0^L\rho(r)\bar k(r)\mathop{\mathrm dr}\approx(8/\pi)Ne^{-L/2}\) and \(\gamma=2\alpha+1\).}
      \item \onslide<3->{\(\theta_i\) is drawn uniformly from \([0,2\pi)\).}
    \end{itemize}
    \item \onslide<4->{For any two nodes \(i,j\), find the hyperbolic distance between them.} \onslide<5->{For large \(L,r_i,r_j\), this is approximately
    \[x_{i,j}=r_i+r_j+2\ln\Bigl(\frac{\min(\lvert\theta_i-\theta_j\rvert,2\pi-\lvert\theta_i-\theta_j\rvert)}2\Bigr)\]}
    \onslide<6->{Connect with the connection probability \(p(x_{i,j})\).} \onslide<7->{The simplest choice is the step function; define \(p(x_{i,j})=1\) for \(x_{i,j}\leq L\) and \(p(x_{i,j})=0\) otherwise.}
  \end{itemize}
\end{frame}

\begin{frame}{Visualizing the model}
  \begin{tabular}{llllll}
    \toprule
    & \(N=20\) & \(N=50\) & \(N=100\) & \(N=200\) & \(N=500\) \\ \midrule
    \(\gamma=2.1\) & \makecell{\input{poincare-model/network_2.1_20.tikz}} & \makecell{\input{poincare-model/network_2.1_50.tikz}} & \makecell{\input{poincare-model/network_2.1_100.tikz}} & \makecell{\input{poincare-model/network_2.1_200.tikz}} & \makecell{\input{poincare-model/network_2.1_500.tikz}} \\
    \(\gamma=2.5\) & \makecell{\input{poincare-model/network_2.5_20.tikz}} & \makecell{\input{poincare-model/network_2.5_50.tikz}} & \makecell{\input{poincare-model/network_2.5_100.tikz}} & \makecell{\input{poincare-model/network_2.5_200.tikz}} & \makecell{\input{poincare-model/network_2.5_500.tikz}} \\
    \(\gamma=3.0\) & \makecell{\input{poincare-model/network_3.0_20.tikz}} & \makecell{\input{poincare-model/network_3.0_50.tikz}} & \makecell{\input{poincare-model/network_3.0_100.tikz}} & \makecell{\input{poincare-model/network_3.0_200.tikz}} & \makecell{\input{poincare-model/network_3.0_500.tikz}} \\ \bottomrule
  \end{tabular}
\end{frame}

\begin{frame}{The result and its converse}
  The resulting network has heterogeneity and clustering.\footcite{Krioukov_2010} \pause

  The converse is also true; a network with heterogeneity and clustering can also be modelled by this model. It is possible to use statistical techniques to infer \((r,\theta)\) for each node using \imp{network embedding}.\footcite{Krioukov_2010} \pause

  Why does this matter? \pause It is possible to use a \imp{greedy algorithm} to route from one node to another. \pause A greedy algorithm makes the locally optimal choice at each stage.
\end{frame}

\section{Routing between nodes}

\begin{frame}{Greedy routing algorithm}
  We will route from node \(m\) to \(n\). Assume we know the coordinates of node \(m\), its neighbours, and its destination \(n\). \pause

  \imp{Original Greedy Forwarding (OGF)}:
  \begin{itemize}
    \item If \(m=n\), the problem is solved. \pause
    \item If, out of all its neighbours, \(m\) itself is the closest to \(n\), it is a local minimum and the algorithm fails. \pause
    \item Otherwise, find the node \(k\) whose hyperbolic distance to \(n\) is least. Route from \(k\) to \(n\). \pause
  \end{itemize}

  \imp{Modified Greedy Forwarding (MGF)}: The algorithm will not exclude the current node. \pause However, it will fail if the best neighbour is the previous node, thus causing an infinite loop. \pause

  With \(\gamma=2.1\), OGF and MGF have a success rate of \(99.92\,\%\) and \(99.99\,\%\), respectively.\footcite{Krioukov_2010}
\end{frame}

\frame[allowframebreaks]{\printbibliography}

\begin{frame}[standout]
  Thank you. Any questions?

  \bigskip

  \qrcode[hyperlink,height=4cm]{https://github.com/danieldanhe/hyperbolic-networks}

  \medskip

  \textmd{Code available on GitHub at \\ \url{https://github.com/danieldanhe/hyperbolic-networks}}
\end{frame}

\end{document}
