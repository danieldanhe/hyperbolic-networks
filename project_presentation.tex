% !TEX TS-program = lualatex

\documentclass[citation=project.bib,garamond,handout]{presentation}

\title{Networks and hyperbolic geometry}
\subtitle{Advanced Topic Post-Euclidean Geometry \\ Project Presentation}
\author{Daniel He \myname}
\institute{Singapore American School}
\date{1 December 2025}
\hypersetup{pdfauthor = {Daniel He 賀守愚}}

\DeclareMathOperator\arcosh{arcosh}
\DeclareMathOperator*{\argmax}{arg\,max}

\usepackage{qrcode}
\usetikzlibrary{external}
\tikzexternalize[prefix=figures/compiled_tikz/]

\begin{document}

\frame{\titlepage}

\begin{frame}[standout]
  Welcome.

  \bigskip

  \qrcode[hyperlink,height=4cm]{https://github.com/danieldanhe/hyperbolic-networks}

  \medskip

  \textmd{Code available on GitHub at \\ \url{https://github.com/danieldanhe/hyperbolic-networks}}
\end{frame}

\frame{\tableofcontents}

\section{Networks}

\begin{frame}{What is a network?}
  A system of interconnected elements (called \imp{nodes} or \imp{vertices}) linked by relationships (called \imp{edges} or \imp{connections}). \pause

  \begin{example}
    \begin{multicols}{3}
      \begin{itemize}[<+->]
        \parskip=0em
        \item Sites on the Internet
        \item Social connections (“six degrees of separation”)
        \item Neural networks and genetic or protein interactions
        \item Road, railway, and air routes
        \item Food webs in ecosystems
        \item Citations among academic papers
        \item International alliances
        \item Finances among banks and institutions
        \item Co-purchases on e-commerce platforms
        \item Microservice dependencies in tech
        \item Transmission of infectious diseases
        \item Semantic networks
        \item Collaborations in arts, music, and film
        \item Blockchain and cryptocurrency transactions
      \end{itemize}
    \end{multicols}
  \end{example}
\end{frame}

\begin{frame}{Properties of these networks}
  What do all of these networks have in common? \pause

  They cannot be modelled by completely random networks, where connections are made by uniform random chance. There are two main differences: \pause
  \begin{itemize}
    \item \imp{Heterogeneity}. \pause Some nodes have more connections than others. \pause On social media, most people have a few hundred followers, but a few celebrities have millions. \pause
    \item \imp{Clustering}. \pause Connections within groups tend to happen more than outside of them. \pause If A and B follow each other, and B and C follow each other, A and C likely follow each other too.
  \end{itemize}
\end{frame}

\begin{frame}{Heterogeneity and clustering}
  \begin{tabular}{lll}
    \toprule
    & Low heterogeneity & High heterogeneity \\ \midrule
    Low clustering & \makecell{\tikzsetnextfilename{low_clust_low_hetero}\input{figures/generate_points/low_clust_low_hetero.tikz}} 
                   & \makecell{\tikzsetnextfilename{low_clust_high_hetero}\input{figures/generate_points/low_clust_high_hetero.tikz}} \\
    High clustering & \makecell{\tikzsetnextfilename{high_clust_low_hetero}\input{figures/generate_points/high_clust_low_hetero.tikz}} 
                    & \makecell{\tikzsetnextfilename{high_clust_high_hetero}\input{figures/generate_points/high_clust_high_hetero.tikz}} \\ \bottomrule
  \end{tabular}

  All networks shown here have approximately \(60\) nodes and \(120\) connections.
\end{frame}

\begin{frame}{Scale-free networks}
  We will look at \imp{scale-free} networks specifically. The degree distribution follows a power law. \pause

  The fraction \(P(k)\) of nodes having \(k\) connections is approximately
  \[P(k)\sim k^{-\gamma}\]
  for large \(k\).\footcite{Barabási_Albert_1999} \pause

  Typically, \(2<\gamma<3\). \pause For the Internet, \(\gamma\approx2.1\).\footcite{Krioukov_2010}
\end{frame}

\begin{frame}{The solution: hyperbolic geometry}
  \begin{tabular}{lll}
    \toprule
    & Low heterogeneity & High heterogeneity \\ \midrule
    Low clustering & \onslide<2->{Random graphs\footcite[Connects pairs completely at random.]{Erdős_Rényi_1959}} & \onslide<3->{Preferential attachment\footcite[New nodes prefer to connect to already well-connected hubs.]{Barabási_Albert_1999}} \\
    High clustering & \onslide<4->{Small-world\footcite[Starts with a regular, clustered lattice and rewires a few connections to create long-range shortcuts.]{Watts_Strogatz_1998}} & \onslide<5->{Hyperbolic geometry} \\ \bottomrule
  \end{tabular}
\end{frame}

\section{Hyperbolic geometry}

\begin{frame}{Distance and area behaviour}
  \onslide<1->{Consider a circle of radius \(r\) on a standard plane.}

  \onslide<2->{
    \begin{tabular}{lll}
      \toprule
      & Circumference of a circle & Area of a circle \\ \midrule
      \onslide<7->{Hyperbolic & \(2\pi\sinh r\onslide<8->{\sim e^r}\) & \(2\pi(\cosh r-1)\onslide<8->{\sim e^r}\)} \\
      \onslide<3->{Euclidean & \(2\pi r\onslide<4->{\sim r}\) & \(\pi r^2\onslide<4->{\sim r^2}\)} \\
      \onslide<5->{Spherical & \(2\pi\sin r\onslide<6->{\to0}\) & \(2\pi(1-\cos r)\onslide<6->{\to4\pi}\)} \\ \bottomrule
    \end{tabular}
  }

  \onslide<9->{Hyperbolic geometry has “more space”.}
\end{frame}

\section{The connection between the two}

\begin{frame}{Network generation in hyperbolic space: Einsteinian model}
  The \imp{Einsteinian-\(\mathbb H^2\) model} generates directly in hyperbolic space. \pause

  There are \(N\) nodes. Each node \(i\) is assigned coordinates \((r_i,\vartheta_i)\) on the hyperbolic plane. \pause
  \begin{itemize}
    \item \(r_i\) is drawn from the probability density function
    \[\rho(r)=\frac{\alpha\sinh(\alpha r)}{\cosh(\alpha R)-1}\sim e^{\alpha r}\]
    where
    \[\alpha=\frac{\gamma-1}2,\quad R\approx2\ln\biggl(\frac{2N}{\pi\bar k}\Bigl(\frac{\gamma-1}{\gamma-2}\Bigr)^2\biggr)\] \pause
    \item \(\vartheta_i\) is drawn uniformly from \([0,2\pi)\).
  \end{itemize}
\end{frame}

\begin{frame}{Network generation in hyperbolic space: Einsteinian model}
  For any two nodes \(i,j\), find the effective distance
  \[x_{i,j}=\arcosh\Bigl(\cosh r_i\cosh r_j-\sinh r_i\sinh r_j\cos\mathrm\Delta\vartheta_{i,j}\Bigr)\approx r_i+r_j+2\ln\Bigl(\frac{\mathrm\Delta\vartheta_{i,j}}2\Bigr)\]
  where
  \[\mathrm\Delta\vartheta_{i,j}=\min(\lvert\vartheta_i-\vartheta_j\rvert,2\pi-\lvert\vartheta_i-\vartheta_j\rvert)\] \pause
  Connect with connection probability
  \[p(x_{i,j})=\frac1{e^{\beta(x_{i,j}-R)/2}+1}\] \pause
  Clustering is maximized when \(\beta\to\infty\); the probability reduces to a step function (connect only if \(x_{i,j}<R\)). \pause The inverse of \(\beta\) is called the \imp{temperature \(T\).}

  We operate under the \imp{cold regime} – that \(\beta>1\).
\end{frame}

\begin{frame}{Visualizing the model}
  \(\bar k=20,\beta=\infty\)

  \begin{tabular}{llllll}
    \toprule
    & \(N=20\) & \(N=50\) & \(N=100\) & \(N=200\) & \(N=500\) \\ \midrule
    \(\gamma=2.1\) 
      & \makecell{\tikzsetnextfilename{gamma2.1_20}\input{figures/poincare_model/network_gamma_2.1_20.tikz}}
      & \makecell{\tikzsetnextfilename{gamma2.1_50}\input{figures/poincare_model/network_gamma_2.1_50.tikz}}
      & \makecell{\tikzsetnextfilename{gamma2.1_100}\input{figures/poincare_model/network_gamma_2.1_100.tikz}}
      & \makecell{\tikzsetnextfilename{gamma2.1_200}\input{figures/poincare_model/network_gamma_2.1_200.tikz}}
      & \makecell{\tikzsetnextfilename{gamma2.1_500}\input{figures/poincare_model/network_gamma_2.1_500.tikz}} \\
    \(\gamma=2.5\) 
      & \makecell{\tikzsetnextfilename{gamma2.5_20}\input{figures/poincare_model/network_gamma_2.5_20.tikz}}
      & \makecell{\tikzsetnextfilename{gamma2.5_50}\input{figures/poincare_model/network_gamma_2.5_50.tikz}}
      & \makecell{\tikzsetnextfilename{gamma2.5_100}\input{figures/poincare_model/network_gamma_2.5_100.tikz}}
      & \makecell{\tikzsetnextfilename{gamma2.5_200}\input{figures/poincare_model/network_gamma_2.5_200.tikz}}
      & \makecell{\tikzsetnextfilename{gamma2.5_500}\input{figures/poincare_model/network_gamma_2.5_500.tikz}} \\
    \(\gamma=3.0\) 
      & \makecell{\tikzsetnextfilename{gamma3.0_20}\input{figures/poincare_model/network_gamma_3.0_20.tikz}}
      & \makecell{\tikzsetnextfilename{gamma3.0_50}\input{figures/poincare_model/network_gamma_3.0_50.tikz}}
      & \makecell{\tikzsetnextfilename{gamma3.0_100}\input{figures/poincare_model/network_gamma_3.0_100.tikz}}
      & \makecell{\tikzsetnextfilename{gamma3.0_200}\input{figures/poincare_model/network_gamma_3.0_200.tikz}}
      & \makecell{\tikzsetnextfilename{gamma3.0_500}\input{figures/poincare_model/network_gamma_3.0_500.tikz}} \\ 
    \bottomrule
  \end{tabular}
\end{frame}

\begin{frame}{The result and its converse}
  The resulting network has heterogeneity and clustering.\footcite{Krioukov_2010} \pause

  The converse is also true; a network with heterogeneity and clustering can also be modelled by this model. It is possible to use statistical techniques to infer \((r,\vartheta)\) for each node using \imp{network embedding}.\footcite{Krioukov_2010} \pause

  Why does this matter? \pause It is possible to use a \imp{greedy algorithm} to route from one node to another. \pause A greedy algorithm makes the locally optimal choice at each stage.
\end{frame}

\section{Routing between nodes}

\begin{frame}{Greedy routing algorithm}
  We will route from node \(m\) to \(n\). Assume we know the coordinates of node \(m\), its neighbours, and its destination \(n\). \pause

  \imp{Original Greedy Forwarding (OGF)}:
  \begin{itemize}
    \item If \(m=n\), the problem is solved. \pause
    \item If, out of all its neighbours, \(m\) itself is the closest to \(n\), it is a local minimum and the algorithm fails. \pause
    \item Otherwise, find the node \(k\) whose hyperbolic distance to \(n\) is least. Route from \(k\) to \(n\). \pause
  \end{itemize}

  \imp{Modified Greedy Forwarding (MGF)}: The algorithm will not exclude the current node. \pause However, it will fail if the best neighbour is the previous node, thus causing an infinite loop. \pause

  With \(\gamma=2.1\), OGF and MGF have a success rate of \(99.92\,\%\) and \(99.99\,\%\), respectively.\footcite{Krioukov_2010}
\end{frame}

\begin{frame}{Bidirectional routing}
  Normal routing typically has time complexity \(\mathcal O(N)\), meaning that if the number of nodes doubles, the number of calculations required also doubles. \pause
  
  While this is not an issue, starting routing from two nodes simultaneously and expanding both searches until they meet is surprisingly fast for hyperbolic networks, amounting to sublinear time. When \(\gamma=2.5\), bidirectional routing is \(\mathcal O(N^{2/3})\).\footcite{Bläsius_2018}
\end{frame}

\section{From data to a visualization}

\begin{frame}{Network generation in hyperbolic space: Newtonian model}
  The \imp{Newtonian-\(\mathbb S^1\) model} generate nodes along a circle and utilizes a hidden variable \imp{expected degree \(\varkappa\)} to calculate distance. \pause

  There are \(N\) nodes. Each node \(i\) is assigned coordinates \((\varkappa_i,\vartheta_i)\). \pause
  \begin{itemize}
    \item \(\varkappa_i\) is drawn from the power-law distribution
    \[\rho(\varkappa)=\varkappa_0^{\gamma-1}(\gamma-1)\varkappa^{-\gamma}\]
    where
    \[\varkappa_0=\frac{\bar k(\gamma-2)}{\gamma-1}\] \pause
    \item \(\vartheta_i\) is drawn uniformly from \([0,2\pi)\).
  \end{itemize}
\end{frame}

\begin{frame}{Network generation in hyperbolic space: Newtonian model}
  For any two nodes \(i,j\), find the hyperbolic distance
  \[\chi_{i,j}=\frac{N\mathrm\Delta\vartheta_{i,j}}{2\pi\mu\varkappa_i\varkappa_j}\]
  where
  \[\mu=\frac\beta{2\pi\bar k\sin(\pi/\beta)}\] \pause
  Connect with connection probability
  \[\tilde p(\chi_{i,j})=\frac1{\chi_{i,j}^\beta+1}\] \pause
  When \(\beta\to\infty\), the probability reduces to a step function (connect only if \(\chi_{i,j}<1\)).
\end{frame}

\begin{frame}{Equivalence of the models}
  It can be proven that the two models are equivalent.\footcite{Boguñá_Papadopoulos_Krioukov_2010} \pause
  
  \(r_i\) and \(\varkappa_i\) map to each other.
  \[\varkappa_i=\varkappa_0e^{(R-r_i)/2}\quad r_i=R-2\ln\frac{\varkappa_i}{\varkappa_0}\] \pause

  The connection probabilities are equivalent.
  \[\chi_{i,j}=e^{(x_{i,j}-R)/2}\]
\end{frame}

\begin{frame}{Network embedding in hyperbolic space: Newtonian model}
  Given a real network with nodes and edges, find coordinates \((r_i,\vartheta_i)\) for each node \(i\). \pause

  The model uses a \imp{maximum likelihood estimator (MLE)} to maximize the \imp{likelihood \(\mathcal L\)}
  \[\mathcal L=\prod_{i<j}p(x_{i,j})^{a_{i,j}}\bigl(1-p(x_{i,j})\bigr)^{1-a_{i,j}}\]
  that the model produces the observed network, where \(a_{i,j}\) is the indicator that nodes \(i\) and \(j\) are connected.
\end{frame}

\begin{frame}{Network embedding in hyperbolic space: Newtonian model}
  \begin{itemize}
    \item \(\varkappa_i^*\) is chosen with
    \[\varkappa_i^*=\max\Bigl(\varkappa_0,k_i-\frac\gamma\beta\Bigr)\] \pause
    \item \(\vartheta_i^*\) is chosen numerically such that it maximizes the overall log-likelihood
    \[\vartheta_i^*=\argmax_{\vartheta_i}\sum_{i<j}\Bigl(a_{i,j}\ln p(x_{i,j})+(1-a_{i,j})\ln\bigl(1-p(x_{i,j})\bigr)\Bigr)\] \pause
  \end{itemize}
  Convert each \(\varkappa_i^*\) to \(r_i^*\).
  \[r_i^*=R-2\ln\frac{\varkappa_i^*}{\varkappa_0}\]
  This is the maximum likelihood estimator that the observed network was generated by the model.\footcite{Boguñá_Papadopoulos_Krioukov_2010}
\end{frame}

\begin{frame}{The practical issue}
  This is quite computationally expensive. The process of generating, even with the Newtonian model, has time complexity \(\mathcal O(N^3)\). \pause

  Therefore, a more practical algorithm would use a layered approach that exploits the properties of scale-free networks.
\end{frame}

\begin{frame}{Layered approach}
  \begin{itemize}
    \item Start with the highest degree nodes; set a boundary and generate a subgraph using the algorithm and numerical optimization with only the nodes with a degree higher than the boundary. \pause
    \item Gradually add nodes with lower degrees. \pause For each new node,
    \begin{itemize}
      \item compute \(r_i\) analytically using the algorithm \pause
      \item if \(i\) has at least \(1\) connection to existing mapped nodes, choose the circular mean
      \[\theta_i=\arctan\Bigl(\frac{\sum_j\sin\theta_j}{\sum_j\cos\theta_j}\Bigr)\quad\text{for connected nodes \(j\)}\] \pause
    \end{itemize}
  \end{itemize}
  This reduces the time complexity to approximately \(\mathcal O(N\log N)\) or even \(\mathcal O(N)\).
\end{frame}

\frame[allowframebreaks]{\printbibliography}

\begin{frame}[standout]
  Thank you. Any questions?

  \bigskip

  \qrcode[hyperlink,height=4cm]{https://github.com/danieldanhe/hyperbolic-networks}

  \medskip

  \textmd{Code available on GitHub at \\ \url{https://github.com/danieldanhe/hyperbolic-networks}}
\end{frame}

\end{document}
